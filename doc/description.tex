\documentclass[12pt]{article}
%	options include 12pt or 11pt or 10pt
%	classes include article, report, book, letter, thesis

\usepackage{amsmath}

\title{A model of competing ideas}
\author{Xavier Rubio-Campillo \\ Iza Romanowska \\ Jonàs Alcaina}
\date{\today}

\begin{document}
\maketitle


\section{Description}

We will create a model depicting competition between two cultural traits within a common population. This is a typical cultural dynamic where individuals must adopt one option amongst two or more mutually exclusive options (i.e. religion, elections, football teams, ...). The assumption of mutual exclusiveness allows us to create a simple model, but more complex versions with individuals adopting more than one trait can easily be developed.

Individuals can change their choice over time. The decision is based on the payoff of each trait. This payoff is a measure of the relative interest of the trait, based on a) how many people exhibits it and b) an internal attractiveness.

An example of this dynamic could be a competition between two different religions. The number of people practicing a belief makes this belief more appealing. However, some beliefs could be intrinsically more interesting for some individuals so part of the population could adopt them even if they are a minority. Finally, beliefs are not static so their attractiveness could vary over time.

\section{Definition}

Our population is defined by a set of $N$ individuals. It is split in two groups $A$ and $B$; $A$ is the number of individuals adopting trait $a$ and $B$ is the group adopting trait $b$. At any given time $t$ an individual is part of $A$ or $B$. Their attractiveness is defined as $Ta$ and $Tb$.

Time is divided in discrete steps starting at $t=0$. On each step $t$ current populations $A_t$ and $B_t$ are updated by 2 quantities: the individuals moving from $A$ to $B$ are defined as $C_{A \to B}$ and the ones moving from $B$ to $A$ defined as $C_{B \to A}$. Equation~\ref{eq1} shows the two equations summarizing this dynamic.

\begin{equation}
\begin{split}
A_{t+1} = A_t + C_{B \to A} - C_{A \to B} \\
B_{t+1} = B_t + C_{A \to B} - C_{B \to A}
\end{split}
\label{eq1}
\end{equation}

The fraction of population switching its trait is proportional to a change rate $\alpha$. As we mentioned is also defined by the number of adopters of the other trait as well as its internal attractiveness. The first term is defined as the proportion of entire population $N$ with this trait ($\frac{A_t}{N}$ for $A_t$ and $\frac{B_t}{N}$ for $B_t$). The last element is the balance between the attractiveness of both ideas. To compute this value we divide the attractiveness of the competing trait by the difference between both of them. The competition equations guiding cultural shift are defined in~\ref{eq2}.

\begin{equation}
\begin{split}
C_{B \to A} = \alpha B_t \left(\frac{A_t}{N}\right) \left(\frac{Ta_t}{Ta_t-Tb_t}\right) \\
C_{A \to B} = \alpha A_t \left(\frac{B_t}{N}\right) \left(\frac{Tb_t}{Tb_t-Ta_t}\right)
\end{split}
\label{eq2}
\end{equation}

Finally, the attractiveness of each trait is a dynamic property of the trait. Each time step $t$ the attractiveness $T$ is slightly modified by an stochastic kernel $K$ as defined in~\ref{eq3}.

\begin{equation}
\begin{split}
T_{A,t+1} = T_{A,t} + K_a \\
T_{B,t+1} = T_{B,t} + K_b \\
\end{split}
\label{eq3}
\end{equation}

$K$ can have several shapes such as:
\begin{itemize}
\item Fixed traits with $K_a=K_b=0$
\item Gaussian process with $K=N(0,1)$
\item A combination (i.e. $K_a=N(0,1)$ and $K_b=t/3$)
\end{itemize}

\end{document}

